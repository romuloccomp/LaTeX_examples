% Artigo para as disciplinas: Heuristica e Dinamica Populacional de células
% CEFET - MMC
% Romulo da Silva Pereira
% romuloccomp@gmail.com

\documentclass[a4paper,12pt]{article}
\usepackage[brazil,english]{babel}
\usepackage[utf8]{inputenc}
\usepackage{graphicx,graphpap}
\usepackage{hyperref}
\usepackage{enumerate}
\usepackage{pgf}
\usepackage{tikz}
\usepackage{algorithmic}
\usepackage{algorithm}

\usetikzlibrary{arrows,automata}

\renewcommand{\labelitemii}{$\bullet$}

\selectlanguage{brazil} 

\date{Feverreiro/2012}

\begin{document}

    %algoritmo
   \begin{algorithm}
     \caption{Multi-Start}
     \label{alg1}
     \begin{algorithmic}[1] 
       \STATE \COMMENT{Realiza interação até a quantidade MAX\_i}
       \WHILE{$i < MAX\_i$}
         \STATE \COMMENT{Cria a distribuição com o número máximo disponível}
         \STATE $taxa \Leftarrow rand\_int (n\_vacinas)$
         \STATE \COMMENT{Executa o MBI com a distribuição gerada}
         \STATE $s \Leftarrow exec\_mbi(taxa)$
         \STATE \COMMENT{Realiza busca nas soluções vizinhas até encontrar um melhor ou MAX\_i}        
         \WHILE{$s < s' and j < MAX\_i$}
           \STATE $taxa' \Leftarrow cria\_solucao\_local (taxa)$
           \STATE $s' \Leftarrow exec\_mbi(taxa')$
         \ENDWHILE
         \STATE \COMMENT{Compara o resultado com a melhor solução encontrada}        
         \IF{$s' < s\_melhor$}
           \STATE \COMMENT{Caso for melhor salva a solução atual}
           \STATE $s\_melhor \Leftarrow s'$
         \ENDIF   
       \ENDWHILE
       \STATE \COMMENT{No final é apresentado a melhor solução encontrada}
     \end{algorithmic}
   \end{algorithm}
  
\end{document}


